{\parskip=10pt

\color{RedOrange}

TO: Mr Snowden

In terms of my A2 Computing Project, I have a basic idea which I think would be workable and fit what you described previously, please let me know if you have any more ideas to expand the program.
The program will feature relatively simple visuals, probably 2D, meaning it should run well on low end computers in the school, the program will simulate individual objects at the appropriate level in sped-up real time.
You will be able to set up scenarios by introducing objects, varying the mass and velocity of the object before placement, multiple objects can be placed and they will interact with each other. (n-body problem could be interesting)

Creating the user interface for this will be somewhat of a challenge, likely I will try to integrate it all into a single window as it would remove the need for a secondary library for the user interface, information about what state an object is going to be placed down in (Mass, Velocity, Size, Fixed) As well as simulation state. (Time, Simulation Speed), A right click menu could contain certain options (Presets) but keyboard controls would the primary method of control.

I have also considered implementing a system which would allow the saving and loading of different scenarios, allowing them to be set-up, saved and loaded at other times, which could be useful in a classroom situation, I would be interested to know if you would find this feature useful.

If you have any questions, let me know, I need to have a bit of a dialogue going in order to establish some groundwork, as the project progresses I will have questions for you in regards to particular decisions or compromises that need to be made along the way.

Many Thanks

Byron Theobald.

\pagebreak

\color{Blue}
TO: Byron

A really useful innovation might be to have a graphical way of representing initial velocity, e.g. an arrow extending from each mass, representing the vector for initial velocity.
I like the idea of being able to save the scenario for later on, it would definitely increase the value of the tool.

A very difficult (but very useful) feature might be to be able to tick a box to display both gravitational field lines and lines of equipotential. This is an area of Physics which you haven’t covered yet, but is fairly straightforward...

Thanks for trying a Physics application!

SDS

}