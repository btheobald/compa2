\paragraph{}
In order to speed up the testing of the program, some basic test scenarios have been defined in order to provide a known system that can be loaded into the application during testing in order to speed up the testing process.
\vspace{12pt}
\begin{enumerate}
\item \textbf{(Ref: TS1)} Single Body with no velocity.
  \begin{itemize}
  \item Should stay completely stationary.
  \item \textbf{IDT:} 0.1 (Simulation Delta Time)
  \item \textbf{UGC:} 0.1 (Gravitational Constant)
  \item \textbf{BC:} 1 (Body Count)
    \begin{itemize}
    \item \textbf{Mass:} 1, \textbf{Radius:} 1, \textbf{Pos X:} 0 \textbf{Y:} 0, \textbf{Vel X:} 0 \textbf{Y:} 0
    \end{itemize}
  \end{itemize}

\vspace{12pt}  
\item \textbf{(Ref: TS2)} Two body orbital system, orbital velocity.
  \begin{itemize}
  \item Outer body has the velocity required for a circular orbit around the central body.
  \item This is calculated using $v_{circular}=\sqrt{\frac{G(M+m)}{r}}$.
  \item Orbital period can be found using $T=\sqrt{\frac{4\pi^2r^3}{Gm}}$.
  \item It should take $628.3$ seconds or $6283$ Iterations to complete one orbit. ($T/IDT$)
  \item \textbf{IDT:} 0.1
  \item \textbf{UGC:} 0.1
  \item \textbf{BC:} 2
    \begin{itemize}
    \item \textbf{Mass:} 1000, \textbf{Radius:} 10, \textbf{Pos X:} 0 \textbf{Y:} 0, \textbf{Vel X:} 0 \textbf{Y:} 0
    \item \textbf{Mass:} 0.1, \textbf{Radius:} 1, \textbf{Pos X:} 100 \textbf{Y:} 0, \textbf{Vel X:} 0 \textbf{Y:} 1.00005
    \end{itemize}
  \end{itemize}
  
\vspace{12pt}  
\item \textbf{(Ref: TS3)} Two body collision test, no gravity.
  \begin{itemize}
  \item Both bodies start with velocity which will lead to a collision.
  \item Gravitational constant is 0 to keep the velocity constant.
  \item Bodies merge according to $p=mv$, $m_n=m_1+m_2$, $v_n=(p_1+p_2)/m_n$.
  \item New radius is calculated by adding together circular areas $\pi r^2$.
  \item \textbf{IDT:} 0.1
  \item \textbf{UGC:} 0.0
  \item \textbf{BC:} 2
    \begin{itemize}
    \item \textbf{Mass:} 1, \textbf{Radius:} 1, \textbf{Pos X:} -10 \textbf{Y:} 0, \textbf{Vel X:} 1 \textbf{Y:} 0
    \item \textbf{Mass:} 1, \textbf{Radius:} 1, \textbf{Pos X:} 10 \textbf{Y:} 0, \textbf{Vel X:} -1 \textbf{Y:} 0
    \end{itemize}
  \end{itemize}
  
\vspace{12pt}  
\item \textbf{(Ref: TS4)} Defines a superstructure.
  \begin{itemize}
  \item Starts the simulation with a superstructure, variable body count.
  \item Relies on a separate variable to set body count.
  \item \textbf{IDT:} 0.1
  \item \textbf{UGC:} 0.1
  \item \textbf{BC:} Variable
    \begin{itemize}
    \item \textbf{Superstructure}
      \begin{itemize}
      \item \textbf{Bodies:} Variable
      \item \textbf{Central Mass:} $\num{1e5}$
      \item \textbf{Outer Mass:} 0.1
      \item \textbf{Central Radius:} 10
      \item \textbf{Outer Radius:} 0.1
      \item \textbf{Central Pos X:} 0
      \item \textbf{Central Pos Y:} 0
      \item \textbf{Central Vel X:} 0
      \item \textbf{Central Vel Y:} 0
      \item \textbf{Central Spacing:} 50
      \item \textbf{System Radius:} 1000
      \end{itemize}
    \end{itemize}
  \end{itemize}
  
\vspace{12pt}
\item \textbf{(Ref: TS5)} Body travels beyond simulation limits.
  \begin{itemize}
  \item Should be destroyed once beyond simulation bounds.
  \item \textbf{IDT:} 0.1 (Simulation Delta Time)
  \item \textbf{UGC:} 0.1 (Gravitational Constant)
  \item \textbf{BC:} 1 (Body Count)
    \begin{itemize}
    \item \textbf{Mass:} 1, \textbf{Radius:} 1, \textbf{Pos X:} $\num{9.999e15}$ \textbf{Y:} 0, \textbf{Vel X:} $\num{1e8}$ \textbf{Y:} 0
    \end{itemize}
  \end{itemize}
  
\vspace{12pt}
\item \textbf{(Ref: TS6)} Body travelling faster than the speed of light.
  \begin{itemize}
  \item Should be destroyed on iteration.
  \item \textbf{IDT:} 0.1
  \item \textbf{UGC:} 0.1 
  \item \textbf{BC:} 1
    \begin{itemize}
    \item \textbf{Mass:} 1, \textbf{Radius:} 1, \textbf{Pos X:} 0 \textbf{Y:} 0, \textbf{Vel X:} $\num{3.00e8}$ \textbf{Y:} 0
    \end{itemize}
  \end{itemize}  
  
\pagebreak
\item \textbf{(Ref: TS7)} Grid of bodies for testing coordinate mapping and rendering precision.
  \begin{itemize}
  \item All bodies are fixed, velocity is ignored and not calculated.
  \item Bodies in center are positioned at 10 and 100 positions respectively.
  \item To the right of the system exists 4 bodies spaced an order of magnitude apart in order to show the loss of rendering precision.
  \item \textbf{IDT:} 0.1
  \item \textbf{UGC:} 0.1 
  \item \textbf{BC:} 20
    \begin{itemize}
    \item \textbf{Mass:} 0.001, \textbf{Radius:} 1, \textbf{Pos X:} 10 \textbf{Y:} 0
    \item \textbf{Mass:} 0.001, \textbf{Radius:} 1, \textbf{Pos X:} -10 \textbf{Y:} 0
    \item \textbf{Mass:} 0.001, \textbf{Radius:} 1, \textbf{Pos X:} 0 \textbf{Y:} 10
    \item \textbf{Mass:} 0.001, \textbf{Radius:} 1, \textbf{Pos X:} 0 \textbf{Y:} -10
    \item \textbf{Mass:} 0.001, \textbf{Radius:} 1, \textbf{Pos X:} -10 \textbf{Y:} -10
    \item \textbf{Mass:} 0.001, \textbf{Radius:} 1, \textbf{Pos X:} 10 \textbf{Y:} -10
    \item \textbf{Mass:} 0.001, \textbf{Radius:} 1, \textbf{Pos X:} -10 \textbf{Y:} 10
    \item \textbf{Mass:} 0.001, \textbf{Radius:} 1, \textbf{Pos X:} 10 \textbf{Y:} 10
    \item \textbf{Mass:} 0.001, \textbf{Radius:} 1, \textbf{Pos X:} 100 \textbf{Y:} 0
    \item \textbf{Mass:} 0.001, \textbf{Radius:} 1, \textbf{Pos X:} -100 \textbf{Y:} 0
    \item \textbf{Mass:} 0.001, \textbf{Radius:} 1, \textbf{Pos X:} 0 \textbf{Y:} 100
    \item \textbf{Mass:} 0.001, \textbf{Radius:} 1, \textbf{Pos X:} 0 \textbf{Y:} -100
    \item \textbf{Mass:} 0.001, \textbf{Radius:} 1, \textbf{Pos X:} -100 \textbf{Y:} -100
    \item \textbf{Mass:} 0.001, \textbf{Radius:} 1, \textbf{Pos X:} 100 \textbf{Y:} -100
    \item \textbf{Mass:} 0.001, \textbf{Radius:} 1, \textbf{Pos X:} -100 \textbf{Y:} 100
    \item \textbf{Mass:} 0.001, \textbf{Radius:} 1, \textbf{Pos X:} 100 \textbf{Y:} 100
    \item \textbf{Mass:} 0.001, \textbf{Radius:} 1, \textbf{Pos X:} $\num{1e5}$ \textbf{Y:} 0
    \item \textbf{Mass:} 0.001, \textbf{Radius:} 1, \textbf{Pos X:} $\num{1e6}$ \textbf{Y:} 0
    \item \textbf{Mass:} 0.001, \textbf{Radius:} 1, \textbf{Pos X:} $\num{1e7}$ \textbf{Y:} 0
    \item \textbf{Mass:} 0.001, \textbf{Radius:} 1, \textbf{Pos X:} $\num{1e8}$ \textbf{Y:} 0
    \end{itemize}
  \end{itemize}
  
  
\pagebreak
\item \textbf{(Ref: TS8)} Three body system, Sun - Planet - Moon.
  \begin{itemize}
  \item An outer body orbits a massive central body.
  \item Another body orbits the outer body.
  \item \textbf{IDT:} 0.1
  \item \textbf{UGC:} 0.1 
  \item \textbf{BC:} 3
    \begin{itemize}
    \item \textbf{Mass:} 200, \textbf{Radius:} 5, \textbf{Pos X:} 0 \textbf{Y:} 0, \textbf{Vel X:} 0 \textbf{Y:} 0
    \item \textbf{Mass:} 1, \textbf{Radius:} 0.5, \textbf{Pos X:} 20 \textbf{Y:} 0, \textbf{Vel X:} 0 \textbf{Y:} 1.00005
    \item \textbf{Mass:} 0.001, \textbf{Radius:} 0.1, \textbf{Pos X:} 20 \textbf{Y:} 1, \textbf{Vel X:} 0.316 \textbf{Y:} 1.00005
    \end{itemize}
  \end{itemize}
  
\vspace{12pt}
\item \textbf{(Ref: TS9)} Simple two body system.
  \begin{itemize}
  \item This system is simple enough to be easy to create a trace table for.
  \item The central body is fixed, making it much faster to run calculations.
  \item A large time-step is used in order to make the numbers easier to handle.
  \item \textbf{IDT:} 2
  \item \textbf{UGC:} 0.1 
  \item \textbf{BC:} 2
    \begin{itemize}
    \item \textbf{Mass:} 99, \textbf{Radius:} 1, \textbf{Pos X:} 0 \textbf{Y:} 0, \textbf{Fixed}
    \item \textbf{Mass:} 1, \textbf{Radius:} 1, \textbf{Pos X:} 10 \textbf{Y:} 0, \textbf{Vel X:} 0 \textbf{Y:} 1
    \end{itemize}
  \end{itemize}  
  
\pagebreak  
\item \textbf{(Ref: TS10)} Defines a simple two body system using a superstructure.
  \begin{itemize}
  \item Creates a superstructure with the generation of a single body constrained to 100.
  \item \textbf{IDT:} 0.1
  \item \textbf{UGC:} 0.1
  \item \textbf{BC:} 2
    \begin{itemize}
    \item \textbf{Superstructure}
      \begin{itemize}
      \item \textbf{Bodies:} 1
      \item \textbf{Central Mass:} $\num{1e5}$
      \item \textbf{Outer Mass:} 0.1
      \item \textbf{Central Radius:} 10
      \item \textbf{Outer Radius:} 0.1
      \item \textbf{Central Pos X:} 0
      \item \textbf{Central Pos Y:} 0
      \item \textbf{Central Vel X:} 0
      \item \textbf{Central Vel Y:} 0
      \item \textbf{Central Spacing:} 100.0
      \item \textbf{System Radius:} 100.001
      \end{itemize}
    \end{itemize}
  \end{itemize}
  
\vspace{12pt}  
\item (Ref: TS11) Stable three body system.
  \begin{itemize}
  \item This system is a three-body system which remains stable. (Unless perturbed.)
  \item The outer bodies have velocity such that they precess around each other.
  \item Their presence applies an equal force to both sides of the central body, thus balancing to 0.
  \item Any change will send the system into chaotic motion.
  \item \textbf{IDT:} 0.01
  \item \textbf{UGC:} 0.1 
  \item \textbf{BC:} 3
    \begin{itemize}
    \item \textbf{Mass:} 1, \textbf{Radius:} 1, \textbf{Pos X:} 0 \textbf{Y:} 0, \textbf{Vel X:} 0 \textbf{Y:} 0
    \item \textbf{Mass:} 1, \textbf{Radius:} 1, \textbf{Pos X:} 10 \textbf{Y:} 0, \textbf{Vel X:} 0 \textbf{Y:} 0.11
    \item \textbf{Mass:} 1, \textbf{Radius:} 1, \textbf{Pos X:} -10 \textbf{Y:} 0, \textbf{Vel X:} 0 \textbf{Y:} -0.11
    \end{itemize}
  \end{itemize}
  
\pagebreak 
\item (Ref: TS12) Many Body System
  \begin{itemize}
  \item This system contains several bodies which will very clearly show the interactions occurring correctly between every body.
  \item \textbf{IDT:} 0.01
  \item \textbf{UGC:} 10
  \item \textbf{BC:} 6
    \begin{itemize}
    \item \textbf{Mass:} 1, \textbf{Radius:} 1, \textbf{Pos X:} 10 \textbf{Y:} 2, \textbf{Vel X:} 0 \textbf{Y:} 0
    \item \textbf{Mass:} 1, \textbf{Radius:} 1, \textbf{Pos X:} -4 \textbf{Y:} 4, \textbf{Vel X:} 0 \textbf{Y:} 0
    \item \textbf{Mass:} 1, \textbf{Radius:} 1, \textbf{Pos X:} -18 \textbf{Y:} 16, \textbf{Vel X:} 0 \textbf{Y:} 0
    \item \textbf{Mass:} 1, \textbf{Radius:} 1, \textbf{Pos X:} -8 \textbf{Y:} -12, \textbf{Vel X:} 0 \textbf{Y:} 0
    \item \textbf{Mass:} 1, \textbf{Radius:} 1, \textbf{Pos X:} 6 \textbf{Y:} -13, \textbf{Vel X:} 0 \textbf{Y:} 0
    \item \textbf{Mass:} 1, \textbf{Radius:} 1, \textbf{Pos X:} 8 \textbf{Y:} 11, \textbf{Vel X:} 0 \textbf{Y:} 0
    \end{itemize}
  \end{itemize}
\end{enumerate}