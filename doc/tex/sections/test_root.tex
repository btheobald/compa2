\subsection{Test Scenarios}
\import{sections/design/}{testscenarios.tex}
\import{sections/design/}{testtables.tex}
\pagebreak
\subsection{Compile Flags}
While there was not enough time remaining to implement any for of automated program testing mode into the program, I decided to make use of compile time flags to make running certain tests easier to run or simply possible to run. Such as selecting the particular test scenario, changing initial state or modifying the operation of the system. These are achived through the use of constant declare conditional statements for the compiler preprocessor in code.

\paragraph{}
These compile flags can be used by compiling the program with the makefile, calling with:
\begin{lstlisting}[language=bash]
  make -j12 CS="-DDEFAULT -DSIMITRS=10"
\end{lstlisting}
\paragraph{}
The \textit{-j12} argument enables parallel compilation, which compiles separate source files at the same time in order to decrease the total time taken to compile the program, in this case it allows for 12 concurrent jobs, this is excessive for the number of files compiled in this program, but it should be set to the number of available concurrent threads on the system the application is being compiled on.

\paragraph{}
Because of the way that the makefile has been written, files that have had no change to their source code will not recompile if the object files still exist, this speeds up the compile time by only compiling the files that need to be recompiled. \\
Because of this however, \textit{make clean} must be run first in order to delete the current executable and all existing object code files before the changes to the compile flags take effect.

\paragraph{}
The following is a list of all of the currently implemented flags that can be defined at compile time to modify the function of the program for testing.\\
\begin{itemize}
\footnotesize
\item \textbf{DEFAULT} - Sets the default start-up scenario.
\item \textbf{TS1} - 12 - Sets the start-up scenario to one of the predefined test scenarios.
\item \textbf{SIMITRS} - Sets a limit to the number of iterations that the simulation will run for. (This does require the program to force close as the simulation thread exits early.)
\item \textbf{SIPF} - Causes an 'i' to be printed to the cout stream on every iteration, not just rendered frames. Newline is printed when frame is sent to shared.
\item \textbf{SCI} - Prints the current iteration since program launch when it is paused.
\item \textbf{PRINTV} - Prints the XY velocity of all bodies every iteration.
\item \textbf{PRINTV} - Prints the XY position of all bodies every iteration.
\item \textbf{PRINTAC} - Prints A and B being calculated in the acceleration calculation. (Relationships being calculated)
\item \textbf{PRINTFORCE} - Prints the result of the force calculated for each relationship. 
\item \textbf{PRINTMXY} - Print current mouse XY coordinates. (Window coordinates)
\item \textbf{PRINTMACT} - Print current mouse XY coordinates on mouse click. (World coordinates)
\item \textbf{EXITNOTE} - Prints notifications for simulation thread exit and main thread exit.
\item \textbf{NOSYNC} - Removes thread synchronisation, allows for the simulation to run much faster than simulation, can cause issues.
\item \textbf{C\_UGC} - Custom gravitational constant.
\item \textbf{C\_IDT} - Custom iteration delta time.
\item \textbf{C\_PAUSED} - Custom paused switch.
\item \textbf{C\_COLLIDE} - Custom collision switch.
\item \textbf{TS4BODIES} - Declares the numbers that will be generated by the TS4 superstructure.
\end{itemize}

\subsubsection{System Tests}
\begin{itemize}
\footnotesize
\item \textbf{SYS1} - DEFAULT
\item \textbf{SYS2} - DEFAULT
\item \textbf{SYS3} - 
\item \textbf{SYS4} - DEFAULT
\item \textbf{SYS5} - EXITNOTE
\item \textbf{SYS6} - DEFAULT
\end{itemize}

\subsubsection{Interface Tests}
\begin{itemize}
\footnotesize
\item \textbf{GUI1} - TS7, C\_UGC=0, PRINTMACT, PRINTMXY
\item \textbf{GUI2} - TS7, C\_UGC=0, PRINTMACT
\item \textbf{GUI3} - TS7, C\_UGC=0, PRINTMACT
\item \textbf{GUI4} - TS7, C\_UGC=0, PRINTMACT
\item \textbf{GUI5} - TS7, C\_UGC=0, PRINTMACT
\item \textbf{GUI6} - 
\item \textbf{GUI7} - PRINTMACT
\item \textbf{GUI8} - 
\item \textbf{GUI9} - 
\item \textbf{GUI10} - DEFAULT
\item \textbf{GUI11} - 
\item \textbf{GUI12} - 
\item \textbf{GUI13} - 
\item \textbf{GUI14} - DEFAULT
\item \textbf{GUI15} - DEFAULT
\item \textbf{GUI16} - TS8, SIPF
\item \textbf{GUI17} - TS7, C\_UGC=0
\end{itemize}

\subsubsection{Simulation Tests}
\begin{itemize}
\footnotesize
\item \textbf{SIM1} - TS1, SIMITRS=10
\item \textbf{SIM2} - TS2, SIMITRS=1
\item \textbf{SIM3} - TS2, SIMITRS=1
\item \textbf{SIM4} - TS2, SIMITRS=6283
\item \textbf{SIM5} - TS2, SIMITRS=6283
\item \textbf{SIM6} - TS3, C\_UGC=0, C\_PAUSED=1
\item \textbf{SIM7} - TS3, C\_UGC=0, C\_PAUSED=1
\item \textbf{SIM8} - TS3, C\_UGC=0, C\_PAUSED=1
\item \textbf{SIM9} - TS3, C\_UGC=0, C\_PAUSED=1
\item \textbf{SIM10} - TS4, TS4BODIES=Variable, C\_COLLIDE=0, C\_PAUSED=1, NOSYNC, SCI
\item \textbf{SIM11} - TS5, C\_PAUSED=1
\item \textbf{SIM12} - TS6, C\_PAUSED=1
\item \textbf{SIM13} - TS9, SIMITRS=8, C\_IDT=2, PRINTV, PRINTP
\item \textbf{SIM14} - TS8, C\_PAUSED=1
\item \textbf{SIM15} - TS11, C\_IDT=0.01, C\_PAUSED=1
\item \textbf{SIM16} - TS11, C\_IDT=0.01, C\_PAUSED=1
\item \textbf{SIM17} - 
\item \textbf{SIM18} - TS11, SIMITRS=1, PRINTAC
\item \textbf{SIM19} - TS12, SIMITRS=1, PRINTAC, PRINTFORCE, C\_UGC=10
\end{itemize}

\pagebreak
\subsection{Test Results}
\subsubsection{System Tests}
\begin{itemize}
\footnotesize
\item \textbf{SYS1} - {\color{Red}Fail}, \textbf{see notes}, {\color{Green}Pass}
\item \textbf{SYS2} - {\color{Green}Pass}
\item \textbf{SYS3} - {\color{Green}Pass}
\item \textbf{SYS4} - {\color{Green}Pass}
\item \textbf{SYS5} - {\color{Green}Pass}
\item \textbf{SYS6} - {\color{Green}Pass}
\end{itemize}

\subsubsection{Interface Tests}
\begin{itemize}
\footnotesize
\item \textbf{GUI1} - {\color{Green}Pass}
\item \textbf{GUI2} - {\color{Green}Pass}
\item \textbf{GUI3} - {\color{Green}Pass}
\item \textbf{GUI4} - {\color{Green}Pass}
\item \textbf{GUI5} - {\color{Green}Pass}
\item \textbf{GUI6} - {\color{Green}Pass}
\item \textbf{GUI7} - {\color{Green}Pass}
\item \textbf{GUI8} - {\color{Green}Pass}
\item \textbf{GUI9} - {\color{Green}Pass}
\item \textbf{GUI10} - {\color{Green}Pass}
\item \textbf{GUI11} - {\color{Green}Pass}
\item \textbf{GUI12} - {\color{Green}Pass}
\item \textbf{GUI13} - {\color{Green}Pass}
\item \textbf{GUI14} - $0$ {\color{Green}Pass}, $10$ {\color{Green}Pass}, $\num{1E3}$ {\color{Green}Pass}, $152$ {\color{Green}Pass}
\item \textbf{GUI15} - $\num{1E18}$ {\color{Green}Pass}, $ABC$ {\color{Green}Pass}, $\num{-1E17}$ {\color{Green}Pass}, $asdfseaq$ {\color{Green}Pass}
\item \textbf{GUI16} - $1$ {\color{Green}Pass}, $2$ {\color{Green}Pass}, $5$ {\color{Green}Pass}, $32$ {\color{Green}Pass},
\item \textbf{GUI17} - {\color{Green}Pass}
\end{itemize}

\subsubsection{Simulation Tests}
\begin{itemize}
\footnotesize
\item \textbf{SIM1} - {\color{Green}Pass}
\item \textbf{SIM2} - {\color{Green}Pass}
\item \textbf{SIM3} - {\color{Green}Pass}
\item \textbf{SIM4} - {\color{Green}Pass}
\item \textbf{SIM5} - {\color{Green}Pass}
\item \textbf{SIM6} - {\color{Green}Pass}
\item \textbf{SIM7} - Mass $[10,1]$ {\color{Green}Pass}, $[20,100]$ {\color{Green}Pass}, $[1000,1000]$ {\color{Green}Pass},
\item \textbf{SIM8} - Velocity $[-1,1]$ {\color{Green}Pass}, $[2,1]$ {\color{Green}Pass}, $[-150,150]$ {\color{Red}Fail}, \textbf{see notes},
\item \textbf{SIM9} - Radius $[10,1]$ {\color{Green}Pass}, $[20,100]$ {\color{Green}Pass}, $[1000,1000]$ {\color{Green}Pass},
\item \textbf{SIM10} - $10, 20, 40, 80, 100, 200, 400, 800, 1000, 2000, 4000, 10000$ {\color{Green}Pass}, \textbf{see notes}
\item \textbf{SIM11} - {\color{Green}Pass}
\item \textbf{SIM12} - {\color{Green}Pass}
\item \textbf{SIM13} - {\color{Green}Pass}
\item \textbf{SIM14} - {\color{Green}Pass}
\item \textbf{SIM15} - {\color{Green}Pass}
\item \textbf{SIM16} - {\color{Green}Pass}
\item \textbf{SIM17} - {\color{Green}Pass}
\item \textbf{SIM18} - {\color{Green}Pass}
\item \textbf{SIM19} - {\color{Green}Pass}
\end{itemize}

\subsection{Test Notes}
\subsubsection{SYS1}
Test SYS1 initially failed due to an incompatibility with the test program Valgrind and C++ call \textit{random\_device}, which is intended to get a random seed every time the application starts instead of using the current time as a seed.
\paragraph{}
For an unknown reason, this code causes Valgrind to crash on the desktop PC that was running the test, but not on a secondary PC. To get around this, the \textit{random\_device} code was removed during all of the tests that involved the use of Valgrind and a static seed was used.
\subsubsection{SIM8}
SIM8 was successful on all but the last test, where the two bodies are moving towards each other with a velocity of 150 each, due to this speed, the bodies end up not registering a collision and continue on past each other.
\paragraph{}
This is due to a number of factors, but it is a context dependant issue and not something that can be easily rectified in the current algorithm.
\paragraph{}
The cause of this issue is related to the velocity, however it is also dependant on the radius of the objects and the current delta time of the simulation, effectively the velocity of the bodies is so great that they end up moving past each other in the space of a single time step, thus their bounds never collide and a collision does not register and calculate.
\paragraph{}
The solution to this is to either increase the radius of the objects or decrease the timestep, in terms of algorithm, the only solution on that end would be to reduce the timestep when bodies are detected as being close together and going too fast to catch the collision. This would require extensive profiling to ensure that this would work efficiently, it may also cause some instability in terms of the simulation due to the way that leapfrog integration works.
\subsubsection{SIM10}
SIM10 was a quantitative benchmark to examine the performance of the simulation and observe the increase in computational time compared to the number of bodies.
\paragraph{}
At each test point, the simulation was allowed to run for 20s and the number of iterations carried out was recorded, 6 repeats were carried out for each test point to provide a better average.\\

\begin{table}[h]
\footnotesize
\centering
\def\arraystretch{1.5}
\begin{tabular}{|c|c|c|c|c|c|c|c|c|} \hline
Bodies & $i_1$ & $i_2$ & $i_3$ & $i_4$ & $i_5$ & $i_6$ & $i_{mean}$ & $i_{1s}$ \\ \hline
    11 & 13300000 & 13600000 & 13400000 & 13400000 & 13600000 & 13000000 & 13433333 &\\ \hline
    21 &  4890000 &  4830000 &  4870000 &  4800000 &  4880000 &  4790000 &  4843333 &\\ \hline
    41 &	1500000	&  1490000 &	1500000 &  1500000 &	1490000 &  1500000 &  1496667 &\\ \hline
    81 &   426000 &   419000 &   419000 &   417000 &   423000 &   417000 &   420167 &\\ \hline
   101 &   271000 &   278000 &   278000 &   272000 &   275000 &   274000 &   274667 &\\ \hline
   201 &    72100 &    71800 &    72000 &    71800 &    71800 &    72100 &    71933 &\\ \hline
   401 &    18400 &    18300 &    18500	&    18700 &    18500 &    18400 &    18467 &\\ \hline
   801 &     4650	&     4690 &     4640 &   	4650 &     4650 &    	4690 &     4662 &\\ \hline
  1001 &     3021	&     3000 &     2980 &   	2970 &     3030 &   	3000 &     3000 &\\ \hline
  2001 &      756	&      751 &	    752 &	     750 &	    748 &	     743 &      750 &\\ \hline
  4001 &      190 &	     188 &	    187 &      188 &	    187 &    	 186 &      188 &\\ \hline
  8001 &       47 &	      46 &       47 &	      46 &       46 &	      47 &       47 &\\ \hline
 10001 &       30 &	      29 &       29 &	      30 &       30 &	      29 &       30 &\\ \hline


\end{tabular}
\end{table}