\begin{table}[H]
\caption{Design Data Dictionary}
\centering
\footnotesize
\def\arraystretch{1.5}
\begin{tabular}{|M{3cm}|M{3cm}|m{8cm}|}
\hline 
\textbf{Data Name} & \textbf{Data Type} & \textbf{Description} \\ \hline
Mass ($m$) & Double (8 B) & The mass of an particular body or system in the simulation, used in calculations for forces and collisions. \\ \hline
Radius ($r$) & Double (8 B) & Measurement of circular objects, specifically half the width, used for sizes of circular bodies, systems or orbits. Used for collision checking and rendering. \\ \hline
Position ($r$) & Double$\times2$ (16 B) & The position in world space of a particular body or object, independent of units, up to users interpretation. Used in simulation. (XY) \\ \hline
Velocity ($v$) & Double$\times2$ (16 B) & The velocity of any object or system in the simulation, change in unit position per unit time. Used in simulation. (XY) \\ \hline
Acceleration ($a$) & Double$\times2$ (16 B) & The instantaneous acceleration acting on a body, change in unit position per unit time per unit time, 2nd derivative or rate of change, calculated using $a=F/m$. (XY) \\ \hline
Force ($F$) & Double$\times2$ (16 B) & The instantaneous force acting on a body, calculated as the relationship between two bodies and summed on individual components (XY), calculated using $\frac{-GMm}{r^3}\hat{r}_{xy}$ \\ \hline
Fixed & Bool (1 B) & This variable will control if a body is simulated in terms of movement, if it is fixed, its neither its position or velocity will update. \\ \hline
Colour & Float$\times3$ (12 B) & This is a array of 3 floating point variables that describes the colour RGB of an object, the American spelling may be used for continuity with libraries. \\ \hline
Body & Object (93 B) & This object structure represents a body, it has mass, radius, position, velocity and can be acted on by force which translates to acceleration, size estimate is 93 Bytes. \\ \hline
Vector & C++ Container & \textit{Vector} is a data structure provided by the C++ STL (Standard Template Library), it acts as a dynamically resizing array, items can be deleted or added and the vector will resize, it can be accessed like a normal array. \\ \hline
Body Store & Vector & The scenarios contain a body storage vector, effectively array that contains every body object in the current scenario, allowing iterative and management of the bodies. \\ \hline
UGC ($G$) & Double & This is a control variable that specifies the \textit{Universal Gravitational Constant}. It is the constant of proportionality between the product of masses and the inverse proportionality of the square of the distance. \\ \hline
\end{tabular}
\end{table} 

\begin{table}[H]
\caption{Design Data Dictionary Cont.}
\centering
\footnotesize
\def\arraystretch{1.5}
\begin{tabular}{|M{3cm}|M{3cm}|m{8cm}|}
\hline 
\textbf{Data Name} & \textbf{Data Type} & \textbf{Description} \\ \hline
IDT ($\Delta t$) & Double & This is a control variable that specifies the \textit{Itteration Delta Time}. This is the graduation of time that the simulation will iterate for. A smaller value results in a slower but more accurate simulation. \\ \hline
IPF & Integer & This is a control variable that specifies the \textit{Itterations Per Frame}. This value describes how many simulation iterations are carried out before sending frame data, this allows a speed-up is the display but retains accuracy. \\ \hline
Paused & Boolean & This is a control variable that specifies if the simulation is running or not. \\ \hline
Scenario & Object & The scenario object will contain a body storage vector as well as all of the relevant control variables that enable the simulation, separate scenarios are allocated for simulation and render, enabling Thread-Local Storage. \\ \hline
Pi ($\pi$)& Constant & Pi is the ratio of a circle's circumference to its diameter, infinitely recursive, it is defined as 3.14159265358979323846 in the C++ standard library. Not used for any simulation calculations. \\ \hline
$$F=\frac{GmM}{r^2}$$ & Equation & Newtons Law of Gravitation, states that particles attract every other particle with a force that is directly proportional to the product of masses but inversely proportional to the square of the distance between them. Meaning this equation gives the force that two particles will exert on each other due to gravity. \\ \hline
$$F=\frac{GmM}{r^3}\hat{r}_{xy}$$ & Equation & This equation is the vector form of the previous, $\hat{r}_{xy}$ is the component distance for the axis the force is being calculated on. The force that is calculated applies to both bodies. \\ \hline
$$F=ma=\frac{\Delta p}{\Delta t}$$ & Equation & This equation describes newtons second law, which states that force is equal to the rate of change of momentum. Rearrange to $a=\frac{F}{m}$ to calculate acceleration due to force. \\ \hline
$$p=mv$$ & Equation & This equation relates mass and velocity and momentum. Momentum will be conserved in collisions as a vector quantity, and thus can be calculated on the X and Y components. \\ \hline
$$A=\pi{r^2}$$ & Equation & This equation is used for calculating the area of a circle, used for adding together areas of bodies during collisions. \\ \hline
\end{tabular}
\end{table} 

\pagebreak