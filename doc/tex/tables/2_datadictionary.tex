\begin{table}[htbp]
\caption{Data Sources and Destinations}
\centering
\footnotesize
\def\arraystretch{1.5}
\begin{tabular}{|M{3cm}|M{3cm}|m{8cm}|}
\hline 
\textbf{Data Name} & \textbf{Data Type} & \textbf{Description} \\
\hline
Mass of Body & Double & The mass of a particular body, every body will have mass. Defined as an objects reluctance to move, allows for calculation of forces, acceleration and potentially collisions. \\
\hline
Size of Body & Double & The graphical size of an object, could be used for calculation of collisions. \\
\hline
Force between Bodies & Double & Direct calculation for force between two bodies. calculated for every body to every other body. (Vector) \\
\hline
Angle of Force & Double & This will contain the angle of the force acting on the body relative to global Y axis. \\
\hline
Force X/Y & Double & Force between bodies resolved to global X/Y axis. \\
\hline
Acceleration X/Y & Double & Acceleration can be calculated using the mass of the body and the X/Y force. \\
\hline
Initial Velocity & Double & An initial velocity that is specified by the user when a body is placed. (Vector) \\
\hline
Angle of Velocity & Double & This contains the angle of the initial velocity of a body, relative to the positive global Y axis. \\
\hline
Velocity X/Y & Double & The initial velocity will be resolved to global X/Y axis, this value is also modified when the change in velocity is calculated using per frame time and Acceleration X/Y. \\
\hline
Position X/Y & Double & Position is first populated with the position that the user places the body at, it is then modified depending on the X/Y Velocity, updating its position, this is also used when resolving forces and velocity, it is also used to calculate the distance between two bodies to calculate the vector force. \\
\hline
Distance between Bodies & Double & The distance between the two bodies in current calculation, calculated using the positions of the two objects being simulated. \\
\hline
Body & Complex Structure ($\approx 120B$) & This will define a particular body, containing all the variables that relate to a particular body, stored in some kind of structure / object. \\
\hline
Gravitational Constant & Double & This defines the gravitational constant which is used to calculate the force that gravity will exert on two bodies, real value is $\num{6.67408e-11} m^{3} kg^{-1} s^{-2}$ However this value will not be particularly useful and should be something larger to allow for smaller scale simulations. The user will be able to modify this themselves. \\
\hline
Time Multiplier & Double & This value will be used when calculating the simulation time per displayed frame to allow the simulation to be run faster or slower than real time \\
\hline
\end{tabular}
\end{table} 

\begin{table}[htbp]
\caption{Data Sources and Destinations Cont.}
\centering
\footnotesize
\def\arraystretch{1.5}
\begin{tabular}{|M{3cm}|M{3cm}|m{8cm}|}
\hline 
\textbf{Data Name} & \textbf{Data Type} & \textbf{Description} \\
\hline
Paused & Boolean & This variable will be a simple TRUE/FALSE describing if the simulation is paused, time will not advance when this is true. \\
\hline
Scenario & Complex Structure & Size is dependant on the number of bodies present in the simulation, Also contains constant variables that affect simulation, namely time multiplier and the gravitational constant, as well as if the simulation is paused, it may be more efficient and safer to have separate scenarios for simulation and rendering, copying the simulation scenario over to render when a frame has finished simulation. \\
\hline
Pi $\pi$ & Constant & Pi is the ratio of a circle's circumference to its diameter, it is defined as $3.14159265358979323846$ as a constant in the C++ cmath library. While Pi is not entirely necessary for the calculations used in the simulation, it may be useful for writing other functions, particularly when converting between Radians and Degrees. \\
\hline
$$F=\frac{Gm_1m_2}{d^2}$$ & Formula & This equation can be used for calculating the force between two bodies due to gravity, F is a vector Force, G is the Gravitational Constant, m is the mass of a body, d is the distance between the bodies. \\
\hline
$$F_y=F\sin{\theta}$$ & Formula & This equation allows the vector Force or Velocity to be resolved to a single axis/dimension. This resolves the y axis. This will be the main computational cost of the program. \\
\hline
$$F_x=F\cos{\theta}$$ & Formula & This equation allows the vector Force or Velocity to be resolved to a single axis/dimension. This resolves the x axis. This will be the main computational cost of the program \\
\hline
$$F=ma$$ & Formula & This equation is one of the first principles of physics, Newton's Second Law, F is force, m is Mass, a is Acceleration. \\
\hline
$$p=mv$$ & Formula & This equation describes how momentum is related to mass and velocity, this equation can be performed on both the x and y axis in the event of a collision between bodies to calculate its final trajectory, momentum is conserved through collisions. \\
\hline
\end{tabular}
\end{table} 